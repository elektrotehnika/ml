\item\points{90} {\bf Компресија методом $k$-средњих вредности}

У овом задатку, биће примењена метода $k$-средњих вредности у сврху компресије слике с губицима кроз ефективно смањење броја боја које се користе.

Биће коришћене датотеке \texttt{src/k\_means/peppers-small.tiff} као и \texttt{src/k\_means/peppers-large.tiff} које представљају слике.

Датотека \texttt{peppers-large.tiff} садржи слику папричица величине 512$\times$512 пиксела чије су боје представљене са 24 бита. Ово значи да за сваки од 262144 пиксела у слици постоје три осмобитне вредности (у опсегу од 0 до 255) које представљају интензитет црвене, зелене и плаве боје за тај пиксел. Праволинијска представа ове дигиталне слике стога заузима око $262144 \times 3 =
786432$ бајтова (при чему један бајт садржи осам бита). Да би се извршила компресија слике, биће коришћена метода $k$-средњих вредности како би се број боја у слици смањио на свега $k=16$ боја. Прецизније речено, сваки пиксел на слици представља тачку у тродимензионалном $(r, g, b)$ простору. Како би се извршила компресија слике са губицима биће извршена кластеризација ових тачака у простору боја у 16 кластера и биће замењена вредност сваког пиксела са њему најближим центроидом.

Пратити упутства у наставку уз напомену и упозорење да неке од операција могу потрајати и по неколико минута чак и на брзим рачунарима!

\begin{enumerate}

  \item\subquestionpoints{70}
\textbf{[Програмерски задатак] Имплементација компресије методом $k$-средњих вредности.}
Најпре \emph{погледајмо} податке. Унутар \texttt{src/k\_means/} директоријума покренути интерактивни Пајтон интерпретатор и у њему укуцати
%
\begin{center}
  \texttt{from matplotlib.image import imread; import matplotlib.pyplot as plt;}
\end{center}
%
а затим покренути \texttt{A = imread(`peppers-large.tiff')}. Сада је \texttt{A} заправо ``тродимензионална матрица'' где су \texttt{A[:,:,0]}, \texttt{A[:,:,1]} и \texttt{A[:,:,2]} дводимензионални низови димензија 512$\times$512 који респективно садрже вредности црвене, зелене и плаве боје за сваки од пиксела. Слика се може приказати у новом прозору уносом \texttt{plt.imshow(A); plt.show()} наредби.

Пошто већа слика има укупно 262144 пиксела па поступак кластеризације може потрајати, векторска квантизација биће урађена на мањој слици. Стога је неопходно поновити претходни поступак за \texttt{peppers-small.tiff} датотеку.

Надаље ће бити имплементирана компресија слике с губицима у датотеци \texttt{src/k\_means/k\_means.py} у којој се налази шаблон. Третирајући $(r, g, b)$ вредности сваког пиксела као елемент $\Re^3$ простора, имплементирати методу $k$-средњих са 16 кластера над вредностима пиксела у мањој слици итерирајући све до конвергенције. У сврху иницијализације, поставити центроиде сваког од кластера на $(r, g, b)$ вредности случајно изабраних пиксела у слици.

Затим у слици \texttt{peppers-large.tiff} заменити $(r, g, b)$ вредности сваког пиксела са вредношћу њој најближег центроида из скупа центроида израчунатих над \texttt{peppers-small.tiff} сликом. Визуелно упоредити новодобијену слику са оригиналном како би се проверила исправност имплементације. \textbf{Укључити у извештај компресовану слику, поред оригиналне.}


\ifnum\solutions=1 {
  \input{k_means/01-k_means_impl_sol}
} \fi


  \item\subquestionpoints{20} \textbf{Фактор компресије.}

Уколико се слика представи са суженим скупом (од 16) боја, колики је (приближно) фактор компресије слике?


\ifnum\solutions=1 {
  \input{k_means/02-compression_factor_sol}
} \fi

\end{enumerate}

