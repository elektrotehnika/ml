\item\subquestionpoints{18} {\bf Програмерски задатак: друга пресликавања својстава}

Уочено је да је неопходан релативно висок степен $k$ како би се дати тренинг скуп података апроксимирао, а разлог томе је што се овај скуп података не може врло добро објаснити (то јест апроксимирати) полиномима ниског степена. Визуализацијом података, може се закључити да се $y$ може добро апроксимирати простопериодичном (синусоидалном) функцијом. Заправо, подаци су генерисани одабирањем функције $y = \sin(x) + \xi$, где је $\xi$ шум са Гаусовом расподелом. Допунити пресликавање својстава $\phi$ тако да укључује и синусну трансформацију као у наставку:

\begin{align}
\phi(x) = \left[\begin{array}{c} 1\\ x \\ x^2\\ \vdots \\x^k \\ \sin(x) \end{array}\right]\in \mathbb{R}^{k+2} \label{eqn:feature-sine}
\end{align}

Са овако допуњеним пресликавањем својстава истренирати различите моделе за вредности $k=0,1,2,3,5,10,20$ и исцртати резултујуће криве хипотезе преко података као и малочас.

Укључити дијаграм као решење овог подзадатка. Упоредити апроксимационе моделе са претходним подзадатком и укратко прокоментарисати приметне разлике у апроксимацији која користи ово пресликавање својстава.

