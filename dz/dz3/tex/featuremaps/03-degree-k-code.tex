\item\subquestionpoints{18} {\bf Програмерски задатак: регресија полинома $k$-тог степена}

Сада се горња идеја проширује на полиноме $k$-тог степена разматрањем $\phi:\mathbb{R}\rightarrow \mathbb{R}^{k+1}$ које је
\begin{align}
\phi(x) = \left[\begin{array}{c} 1\\ x \\ x^2\\ \vdots \\x^k \end{array}\right]\in \mathbb{R}^{k+1} \label{eqn:feature-k}
\end{align}

Пратити исти поступак као у претходном подзадатку и имплементирати алгоритам за $k=3,5,10,20$. Направити сличне дијаграме као у претходном подзадатку и исцртати криве хипотезе за сваку вредност $k$ користећи различиту боју. Укључити и натписе у дијаграму да се укаже која боја одговара којој вредности $k$

Укључити дијаграм у извештај као решење овог подзадатка. Посматрати како се апроксимација тренинг скупа мења када се $k$ повећава. Укратко прокоментарисати запажања.

