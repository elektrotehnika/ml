\item \subquestionpoints{30} \textbf{[Програмерски подзадатак] Проблем коктел журке}

У овом подзадатку биће имплементиран АНК алгоритам претпостављајући Лапласове изворе (као што је изведено у претходном подзадатку) уместо Логистичких извора који су обрађени на предавањима. Датотека \texttt{src/ica/mix.dat} садржи улазне податке који се састоје из матрице са пет колона, где свака колона одговара једном измешаном сигналу $x_i$. Шаблон изворног кода за овај подзадатак налази се у \texttt{src/ica/ica.py} где треба допунити \texttt{update\_W} и \texttt{unmix} функције.

Након тога може се покренути \texttt{ica.py} да се измешани аудио запис раздвоји на компоненте. Измешани аудио записи ће бити снимљени у \texttt{mixed\_i.wav} у излазном директоријуму. Раздвојени аудио записи ће бити снимљени у \texttt{split\_i.wav} у излазном директоријуму.

Како би се уверили да је решење исправно, преслушати раздвојене аудио записе. (Неко преклапање или шум у изворима може бити присутан, али различити извори би требало да буду јасно раздвојени.)

\textbf{Послати матрицу размешавања $W$ (5$\times$5) која је добијена тако што ће и датотека \texttt{W.txt} са овим вредностима бити укључена поред изворног кода.}

У исправној имплементацији, излаз \texttt{split\_0.wav} би требало да звучи јако слично запису у \texttt{correct\_split\_0.wav} који је укључен ради провере.

Напомена: Стопа учења $\alpha$ може се постепено смањивати како би се убрзало тренирање, односно учење. Поред променљиве стопе учења која може убрзати конвергенцију, могуће је такође изабрати случајну пермутацију тренинг података и покренути стохастички градијентни успон по том редоследу над њима.

