\item \subquestionpoints{10}
Подсетити се да је у ГДА заједничка расподела $(x, y)$ описана следећим једначинама:
%
\begin{eqnarray*}
	p(y) &=& \begin{cases}
	\phi & \mbox{if~} y = 1 \\
	1 - \phi & \mbox{if~} y = 0 \end{cases} \\
	p(x | y=0) &=& \frac{1}{(2\pi)^{\di/2} |\Sigma|^{1/2}}
		\exp\left(-\frac{1}{2}(x-\mu_{0})^T \Sigma^{-1} (x-\mu_{0})\right) \\
	p(x | y=1) &=& \frac{1}{(2\pi)^{\di/2} |\Sigma|^{1/2}}
		\exp\left(-\frac{1}{2}(x-\mu_1)^T \Sigma^{-1} (x-\mu_1) \right),
\end{eqnarray*}
%
где су $\phi$, $\mu_0$, $\mu_1$, и $\Sigma$ параметри модела.

Претпоставимо да су $\phi$, $\mu_0$, $\mu_1$, и $\Sigma$ већ одређени и да је даље неопходно предвидети $y$ за нову задату тачку $x$. Како би се доказало да ГДА као резултат даје класификатор са линеарном границом одлуке, показати да се апостериорна вероватноћа може написати као
%
\begin{equation*}
	p(y = 1\mid x; \phi, \mu_0, \mu_1, \Sigma)
	= \frac{1}{1 + \exp(-(\theta^T x + \theta_0))},
\end{equation*}
%
где су $\theta\in\Re^\di$ и $\theta_{0}\in\Re$ одговарајуће функције параметара $\phi$, $\Sigma$, $\mu_0$, и $\mu_1$.

